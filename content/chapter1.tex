\chapter{Preliminaries to Complex Analysis}

\section{Complex number and the complex plane}

\subsection{Basic properties}

% In the subsection, we can learn some basic properties about 

\subsection{Convergence}

% 复数域的柯西收敛准则
\begin{theorem}
    $\mathbb{C}$, the complex numbers, is complete.
\end{theorem}
\begin{proof}
    For a Cauchy sequence of complex numbers $\{z_n\}$, then
    \[
        |z_n-z_m|\to 0 \quad\text{as}\quad n,m\to\infty.
    \]
    In other words, given $\epsilon>0$ there exists an
    integer $N>0$ so that $|z_n-z_m|<\epsilon$ whenever $n,m>N$.
    If assuming $z_n=x_n+iy_n, z_m=x_m+iy_m$, so we can get
    \[
        |z_n-z_m|=\sqrt{(x_n-x_m)^2+(y_n-y_m)^2}.
    \]
    According to Cauchy's convergence theorem: every Cauchy sequence
    of real numbers converges to a real number. So we can get the Cauchy's
    convergence theorem of complex numbers.
\end{proof}

\begin{theorem}
    The set $\Omega\subset \mathbb{C}$ is compact if and only if every sequence
    $\{z_n\}\subset \Omega$ has a subsequence that converges to a point in $\Omega$.
\end{theorem}
\begin{proof}
	For a compact set $\Omega$, then it is closed and bounded.
\end{proof}

\begin{theorem}
    A set $\Omega$ is compact if and only if every open covering of
    $\Omega$ has a finite subcovering.
\end{theorem}
\begin{proof}
\end{proof}

% 复数域的闭区间套定理
\begin{proposition}
    if $\Omega_1\supset\Omega_2\supset\cdots\supset\Omega_n\supset\cdots$ is a sequence
    of non-empty compact sets in $\mathbb{C}$ with the property that
    \[
        \text{diam}(\Omega_n)\to 0\quad\text{as}\; n\to\infty,
    \]
    then there exists a unique point $w\in\mathbb{C}$ such
    that $w\in\Omega_n$ for all $n$.
\end{proposition}
\begin{proof}
    Choose a point $z_n$ in each $\Omega_n$. We prove $\{z_n\}$ is a Cauchy sequence.
    Because of the condition diam$(\Omega_n)\to 0$, so we can get
    \[
        \forall \epsilon>0, \exists N\Rightarrow \text{diam}(\Omega_n)<\epsilon.
    \]
    We take two integers $m,n>N$, so $z_m,z_n\in\Omega_N$. We can get
    \[
        |z_n-z_m|\le \text{diam}(\Omega_n)<\epsilon.
    \]
    $\{z_n\}$ is a Cauchy sequence, therefore this sequence converges to a limit that
    we call $w$. Next, we will prove $w\in \Omega_n$ for all $n$. Finally, $w$ is the
    unique point satisfying this property, for otherwise, if $w'$ satisfied the same
    property with $w'\ne w$ we would have $|w-w'|>0$ and the
    condition diam$(\Omega_n)\to 0$ would be violated.
\end{proof}

\section{Functions on the complex plane}

\subsection{Continuous functions}

\begin{theorem}
    A continuous function on a compact set $\Omega$ is bounded and attains
    a maximum and minimum on $\Omega$.
\end{theorem}
\begin{proof}
    
\end{proof}

\begin{proposition}
    if $f$ and $g$ are holomorphic in $\Omega$, then:
    \begin{enumerate}[(i)]
    \item $f+g$ is holomorphic in $\Omega$ and $(f+g)'=f'+g'$.
    \item $fg$ is holomorphic in $\Omega$ and $(fg)'=f'g+fg'$.
    \item If $g(z_0)\ne 0$, then $f/g$ is holomorphic at $z_0$ and
    \[
        (f/g)'=\frac{f'g-fg'}{g^2}.
    \]
    \end{enumerate}
    Moreover, if $f:\Omega\to U$ and $g:U\to\mathbb{C}$ are holomorphic, the chain
    rule holds
    \[
        (g\circ f)'(z)=g'(f(z))f'(z) \text{for all} z\in\Omega.
    \]
    
    \begin{proof}
        The three are easy and simple. Next, we will prove the chain rule.
        Let $F(z) = (g\circ f)(z)$, so
        \[
            F'(z)=\lim_{h\to 0}\frac{F(z+h)-F(z)}{h}.
        \]
    \end{proof}
\end{proposition}

\begin{proposition}
    If $f$ is holomorphic at $z_0$, then
    \[
        \frac{\partial f}{\partial \bar{z}}(z_0)=0\pard \text{and}
    \]
\end{proposition}

\subsection{Power series}

\begin{theorem}
    Given a power series $\sum_{n=0}^{\infty}a_n z^n$, there exists
    $0\le R\le \infty$ such that:
    \begin{enumerate}[(i)]
        \item If $|z|<R$ the series converges absolutely.
        \item If $|z|>R$ the series diverges.
    \end{enumerate}
    Moreover, if we use the convention that $1/0=\infty$ and $1/\infty=0$, then
    $R$ is given by Hadamard's formula
    \[
        1/R=\lim \sup |a_n|^{1/n}.
    \]
    The number $R$ is called the radius of convergence of the power series,
    and the region $|z|<R$ the disc of convergence. In particular, we have
    $R=\infty$ in the case of the exponential function, and $R=1$ for the
    geometric series.
\end{theorem}
\begin{proof}
\endu{proof}
